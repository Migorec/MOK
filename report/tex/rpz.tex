%% Преамбула TeX-файла

% 1. Стиль и язык
\documentclass[utf8x, 12pt]{G7-32} % Стиль (по умолчанию будет 14pt)

% Остальные стандартные настройки убраны в preamble-std.tex
\sloppy

% 1. Настройки стиля ГОСТ 7-32
% Для начала определяем, хотим мы или нет, чтобы рисунки и таблицы нумеровались в пределах раздела, или нам нужна сквозная нумерация.
% А не забыл ли автор букву 't' ?
%\EqInChapter % формулы будут нумероваться в пределах раздела
%\TableInChapter % таблицы будут нумероваться в пределах раздела
%\PicInChapter % рисунки будут нумероваться в пределах раздела



% 2. Добавляем гипертекстовое оглавление в PDF
\usepackage[
bookmarks=true, colorlinks=true, unicode=true,
urlcolor=black,linkcolor=black, anchorcolor=black,
citecolor=black, menucolor=black, filecolor=black,
]{hyperref}

% 3. Изменение начертания шрифта --- после чего выглядит таймсоподобно.
% apt-get install scalable-cyrfonts-tex

\IfFileExists{cyrtimes.sty}
    {
        \usepackage{cyrtimespatched}
    }
    {
        % А если Times нету, то будет CM...
    }


% 4. Прочие полезные пакеты.
\usepackage{underscore} % Ура! Теперь можно писать подчёркивание.
                        % И нельзя использовать подчёркивание в файлах.
                        % Выбирай, но осторожно.

\usepackage{graphicx}   % Пакет для включения рисунков

 % 5. Любимые команды
\newcommand{\Code}[1]{\textbf{#1}}

% 6. Поля
% С такими оно полями оно работает по-умолчанию:
% \RequirePackage[left=20mm,right=10mm,top=20mm,bottom=20mm,headsep=0pt]{geometry}
% Если вас тошнит от поля в 10мм --- увеличивайте до 20-ти, ну и про переплёт не забывайте:
\geometry{right=20mm}
\geometry{left=30mm}


% 7. Tikz
\usepackage{tikz}
\usetikzlibrary{arrows,positioning,shadows}

% 8 Листинги

\usepackage{listings}

% Значения по умолчанию
\lstset{
  basicstyle= \footnotesize,
  breakatwhitespace=true,% разрыв строк только на whitespacce
  breaklines=true,       % переносить длинные строки
%   captionpos=b,          % подписи снизу -- вроде не надо
  inputencoding=koi8-r,
  numbers=left,          % нумерация слева
  numberstyle=\footnotesize,
  showspaces=false,      % показывать пробелы подчеркиваниями -- идиотизм 70-х годов
  showstringspaces=false,
  showtabs=false,        % и табы тоже
  stepnumber=1,
  tabsize=4,              % кому нужны табы по 8 символов?
  frame=single
}

% Стиль для псевдокода: строчки обычно короткие, поэтому размер шрифта побольше
\lstdefinestyle{pseudocode}{
  basicstyle=\small,
  keywordstyle=\color{black}\bfseries\underbar,
  language=Pseudocode,
  numberstyle=\footnotesize,
  commentstyle=\footnotesize\it
}

% Стиль для обычного кода: маленький шрифт
\lstdefinestyle{realcode}{
  basicstyle=\scriptsize,
  numberstyle=\footnotesize
}

% Стиль для коротких кусков обычного кода: средний шрифт
\lstdefinestyle{simplecode}{
  basicstyle=\footnotesize,
  numberstyle=\footnotesize
}

% Стиль для BNF
\lstdefinestyle{grammar}{
  basicstyle=\footnotesize,
  numberstyle=\footnotesize,
  stringstyle=\bfseries\ttfamily,
  language=BNF
}

% Определим свой язык для написания псевдокодов на основе Python
\lstdefinelanguage[]{Pseudocode}[]{Python}{
  morekeywords={each,empty,wait,do},% ключевые слова добавлять сюда
  morecomment=[s]{\{}{\}},% комменты {а-ля Pascal} смотрятся нагляднее
  literate=% а сюда добавлять операторы, которые хотите отображать как мат. символы
    {->}{\ensuremath{$\rightarrow$}~}2%
    {<-}{\ensuremath{$\leftarrow$}~}2%
    {:=}{\ensuremath{$\leftarrow$}~}2%
    {<--}{\ensuremath{$\Longleftarrow$}~}2%
}[keywords,comments]

% Свой язык для задания грамматик в BNF
\lstdefinelanguage[]{BNF}[]{}{
  morekeywords={},
  morecomment=[s]{@}{@},
  morestring=[b]",%
  literate=%
    {->}{\ensuremath{$\rightarrow$}~}2%
    {*}{\ensuremath{$^*$}~}2%
    {+}{\ensuremath{$^+$}~}2%
    {|}{\ensuremath{$|$}~}2%
}[keywords,comments,strings]

% Подписи к листингам на русском языке.
\renewcommand*\thelstnumber{\oldstylenums{\the\value{lstnumber}}}
\renewcommand\lstlistingname{\cyr\CYRL\cyri\cyrs\cyrt\cyri\cyrn\cyrg}
\renewcommand\lstlistlistingname{\cyr\CYRL\cyri\cyrs\cyrt\cyri\cyrn\cyrg\cyri}

% Произвольная нумерация списков.
\usepackage{enumerate}



\begin{document}

\frontmatter % выключает нумерацию ВСЕГО; здесь начинаются ненумерованные главы: реферат, введение, глоссарий, сокращения и прочее

% Команды \breakingbeforechapters и \nonbreakingbeforechapters
% управляют разрывом страницы перед главами.
% По-умолчанию страница разрывается.

\nobreakingbeforechapters
% \breakingbeforechapters


\begin{center}
Отчет по лабораторной работе \\ по курсу \textit{Модели оценки качества} \\ Вариант 17-12 
\end{center}
\begin{flushright}
студент: Миникс И.В.\\ группа: ИУ7-101
\end{flushright}

\section{Формулировка задачи}

В ВС, содержащей N процессоров и M каналов обмена данными, постоянно находятся K задач. Разработать модель (исходную и с применением укрупнения моделей), оценивающую производительность системы с учетом отказов и восстановлений процессоров и каналов. Имеется не более L ремонтных бригад, которые ремонтируют отказывающие устройства с бесприоритетной дисциплиной. 

Сравнить полученные результаты с результатами имитационного моделирования. Интенсивность отказов, восстановлений, средние времена обработки сообщения и среднее время обдумывания известны и задаются пользователем. Известно также количество процессоров, каналов и задач.

Построить графики зависимостей выходных результатов от исходных данных. С помощью имитационной модели исследовать влияние функций распределения интервалов времени прихода требований и времени обслуживания на выходные результаты.


\section {Исходная модель}
На рисунке~\ref{fig:main} представлена схема исходной модели системы.
\begin{figure}[ht]
\centering
\includegraphics[width=\textwidth]{inc/dia/main}
\caption{Исходная модель}
\label{fig:main}
\end{figure}

\newpage
\subsection {Параметры системы}
\begin {enumerate}[]
\item K --- число задач.
\item M --- число процессоров.
\item N --- число каналов.
\item L --- число ремонтных бригад.
\item $\lambda$ --- интенсивность обдумывания.
\item $\mu$ --- интенсивность обработки задачи на процессорной фазе.
\item $\nu$ -- интенсивность обработки задачи на канальной фазе.
\item $\alpha$ --- интенсивность отказа процессоров.
\item $\beta$ --- интенсивность восстановления процессоров.
\item $\gamma$ --- интенсивность отказа каналов.
\item $\delta$ --- интенсивность восстановления каналов.
\end{enumerate}



\section{Моделирование отказов и восстановлений}
Состояние системы будем описывать вектором $ \xi (t) = (\xi_{1}(t),\,\xi_{2}(t))$, где $\xi_{1}(t)$ - число неисправных процессоров в момент времени $t$, $\xi_{2}(t)$ - число неисправных каналов в момент времени $t$.

На рисунке~\ref{fig:broke-graph} показана структура фрагмента графа состояний системы, где $\beta_{ij}=\beta\frac{i}{i+j}min\left\lbrace i+j,L\right\rbrace$, $\delta_{ij}=\delta\frac{j}{i+j}min\left\lbrace i+j,L\right\rbrace$. 
\begin{figure}[ht]
\centering
\includegraphics[height=5cm]{inc/dia/broke-graph}
\caption{Структура фрагмента графа состояний системы}
\label{fig:broke-graph}
\end{figure}

Проведем укрупнение состояний сисетмы. Объединим в одно макросостояние все врешины графа, у которых одинаковым является первый компонент $\xi_{1}(t)$~--- число неисправных процессоров. Полученный граф представлен на рисунке~\ref{fig:broke-proc}, где $\beta_i=\beta\sum\limits_{j=0}^N\pi_j\frac{i}{i+j}min\left\lbrace i+j,L\right\rbrace$, $\pi_j$~--- вероятность того, что отказали ровно j каналов.

\begin{figure}[ht]
\centering
\includegraphics[width=\textwidth]{inc/dia/broke-proc}
\caption{Граф состояний системы}
\label{fig:broke-proc}
\end{figure}

Тогда выражения для определения вероятностей стационарных состояний примут вид:

\begin{equation}
\label{eq:broke-proc}
\left\{
   \begin{array}{lcl}
	p_{0} = \left( 1 + \dfrac{M \alpha}{\beta_1} +  ... + \dfrac{M! \alpha^{M}}{\prod \limits_{i=1}^M \beta_i} \right) ^{-1} \\
	p_{i} = p_{0} \dfrac{\alpha^{i}\prod \limits_{j=1}^{i} (M-j+1)}{\prod \limits_{j=1}^i \beta_{j}}, \quad i = \overline{1,M}  \\ 
	\beta_i=\beta\sum\limits_{j=0}^N\pi_j\frac{i}{i+j}min\left\lbrace i+j,L\right\rbrace
   \end{array}
\right.
\end{equation}
 
Аналогичным образом объединим в одно макросостояние все врешины графа, у которых одинаковым является второй компонент $\xi_{2}(t)$~--- число неисправных каналов. Полученный граф представлен на рисунке~\ref{fig:broke-chan}, а выражения для определения вероятностей стационарных состояний примут вид:

\begin{equation}
\label{eq:broke-chan}
\left\{
   \begin{array}{lcl}
	\pi_{0} = \left( 1 + \dfrac{N \gamma}{\delta_1} +  ... + \dfrac{N! \gamma^{N}}{\prod \limits_{i=1}^N \delta_i} \right) ^{-1} \\
	\pi_{i} = \pi_{0} \dfrac{\gamma^{i}\prod \limits_{j=1}^{i} (N-j+1)}{\prod \limits_{j=1}^i \delta_{j}}, \quad i = \overline{1,N}  \\ 
	\delta_j=\delta\sum\limits_{i=0}^M p_i\frac{j}{i+j}min\left\lbrace i+j,L\right\rbrace
   \end{array}
\right.
\end{equation}

\begin{figure}[ht]
\centering
\includegraphics[width=\textwidth]{inc/dia/broke-chan}
\caption{Граф состояний системы}
\label{fig:broke-chan}
\end{figure}

Применяя формулы~\ref{eq:broke-proc} и~\ref{eq:broke-chan} итеративно получим вероятности отказов процессоров и каналов в системе. В качестве начального приближения можно взять $\pi_i=\frac{1}{M}$

\newpage

\section{Укрупнение модели}

Заменим исходную модель агрегированной однофазной моделью АМ1 (см. рисунок~\ref{fig:AM1}). В агрегированный узел объединена подсистема, включающая в себя процессоры и каналы. Интенсивность обслуживания в этом узле зависит от числа находящихся в нем заявок.

\begin{figure}[ht]
\centering
\includegraphics[height=5cm]{inc/dia/AM1}
\caption{Укрупненная модель АМ1}
\label{fig:AM1}
\end{figure}

Граф состояний полученной системы представлен на рисунке~\ref{fig:graphAM1}. Производительность системы может быть вычислена по формулам:


\begin{equation}
\label{eq:AM1}
\left\{
   \begin{array}{lcl}
	\hat{\pi}_{0} = \left( 1 + \dfrac{K \lambda}{\xi_1} +  ... + \dfrac{K! \lambda^{K}}{\prod \limits_{i=1}^K \xi_i} \right) ^{-1} \\
	\hat{\pi}_{i} = \hat{\pi}_{0} \dfrac{\lambda^{i}\prod \limits_{j=1}^{i} (K-j+1)}{\prod \limits_{j=1}^i \xi_{j}}, \quad i = \overline{1,K}  \\ 
	\xi_{ср}^{*}=\sum \limits_{i=1}^K \xi_i \hat{\pi_i}
   \end{array}
\right.
\end{equation}


\begin{figure}[ht]
\centering
\includegraphics[width=\textwidth]{inc/dia/graphAM1}
\caption{Граф состояний модели АМ1}
\label{fig:graphAM1}
\end{figure}

Однако, чтобы воспользоваться приведенными формулами, необходимо знать параметры связи $\mu_i$. Чтобы их найти, рассмотрим укрупненную модель АМ2, структура и граф состояний которой показаны на рисунках~\ref{fig:AM2} и~\ref{fig:graphAM2}. За состояние системы примем количество заявок на процессорной фазе, а интенсивности переходов могут быть выражены по формулам:

\begin{equation}
\label{eq:mu}
\mu_i = \mu \sum \limits_{j=0}^M p_j min \left\lbrace i, M-j \right\rbrace
\end{equation}

\begin{equation}
\label{eq:nu}
\nu_i = \nu \sum \limits_{j=0}^N \pi_j min \left\lbrace n-i+1, N-j \right\rbrace
\end{equation}



\begin{figure}[ht]
\centering
\includegraphics[height=4.5cm]{inc/dia/AM2}
\caption{Укрупненная модель АМ2}
\label{fig:AM2}
\end{figure}

\begin{figure}[ht]
\centering
\includegraphics[width=\textwidth]{inc/dia/graphAM2}
\caption{Граф состояний модели АМ2}
\label{fig:graphAM2}
\end{figure}


Параметр связи может вычислен по следующим формулам:


\begin{equation}
\label{eq:AM2}
\left\{
   \begin{array}{lcl}
	\hat{p}_{0} = \left( 1 + \dfrac{\nu_1}{\mu_1} +  ... + \dfrac{\prod \limits_{i=1}^n \nu_i}{\prod \limits_{i=1}^n \mu_i} \right) ^{-1} \\
	\hat{p}_{i} = \hat{p}_{0} \dfrac{\prod \limits_{j=1}^{i} (\nu_j)}{\prod \limits_{j=1}^i \mu_{j}}, \quad i = \overline{1,n}  \\ 
	\xi_n = \sum \limits_{i=1}^n \hat{p}_i \mu_i
   \end{array}
\right.
\end{equation}


\section{Окончательная расчетная схема}
Последовательность расчета производительности системы должна быть следующей:

\begin{enumerate}
\item По формулам~\ref{eq:broke-proc} и ~\ref{eq:broke-chan} вычислить $\pi_i, i=\overline{0,N}$ и $p_i, i=\overline{0,M} $.
\item По формулам~\ref{eq:mu},~\ref{eq:nu} и~\ref{eq:AM2} вычислить $\xi_n, n=\overline{1,K}$.
\item Вычислить $\xi^{*}$ по формулам~\ref{eq:AM1}.
\end{enumerate}


\section{Сравнение теоретической и имитационной модели}

\subsection{Опыт 1}

Условия:
M=1, N=1, L=1, $\lambda = 1$, $\mu=5$, $\nu=5$, $\alpha=\beta=0$. 

$K=\overline{1,50}$

\begin{figure}[ht]
\centering
\includegraphics[height=8cm]{inc/exp/exp1}
\caption{Опыт 1}
\label{fig:exp1}
\end{figure}


\subsection{Опыт 2}

Условия:
K=10, N=10, L=1, $\lambda = 1$, $\mu=5$, $\nu=5$, $\alpha=\beta=0$. 

$M=\overline{1,25}$

\begin{figure}[ht]
\centering
\includegraphics[height=8cm]{inc/exp/exp2}
\caption{Опыт 2}
\label{fig:exp2}
\end{figure}

\subsection{Опыт 3}

Условия:
K=10, M=4, N=4, L=1, $\lambda = 5$, $\mu=20$, $\alpha=\beta=0$. 

$\nu=\overline{5,30}$

\begin{figure}[ht]
\centering
\includegraphics[height=8cm]{inc/exp/exp3}
\caption{Опыт 3}
\label{fig:exp3}
\end{figure}

\subsection{Опыт 4}

Условия:
K=10, M=4, N=4, L=5, $\lambda = 1$, $\mu=3$, $\nu=3$, $\beta=10$, $\gamma=0$. 

$\alpha=\overline{5,25}$

\begin{figure}[ht]
\centering
\includegraphics[height=8cm]{inc/exp/exp4}
\caption{Опыт 4}
\label{fig:exp4}
\end{figure}


\mainmatter % это включает нумерацию глав и секций в документе ниже


\backmatter %% Здесь заканчивается нумерованная часть документа и начинаются ссылки и
            %% заключение



\appendix   % Тут идут приложения


\end{document}

%%% Local Variables:
%%% mode: latex
%%% TeX-master: t
%%% End:
